\documentclass[11pt, a4paper]{article}
\usepackage{sectsty}
\usepackage{enumitem}
\usepackage{graphicx}
\usepackage{amsmath}
\usepackage{amssymb}
\usepackage{setspace}
\usepackage{tasks}
\usepackage{graphicx}
\usepackage{float}
\usepackage{comment}
\usepackage{listings}
\usepackage[utf8]{inputenc}
\usepackage{amsfonts}
\usepackage{gensymb}
\usepackage{multicol}
\usepackage{tabularx}
\usepackage{tikz}
\newcommand{\myvec}[1]{\ensuremath{\begin{pmatrix}#1\end{pmatrix}}}
\let\vec\mathbf

\newcommand{\mydet}[1]{\ensuremath{\begin{vmatrix}#1\end{vmatrix}}}
\providecommand{\brak}[1]{\ensuremath{\left(#1\right)}}
\providecommand{\lbrak}[1]{\ensuremath{\left(#1\right.}}
\providecommand{\rbrak}[1]{\ensuremath{\left.#1\right)}}
\providecommand{\cbrak}[1]{\ensuremath{\left\{#1\right\}}}
\providecommand{\sbrak}[1]{\ensuremath{{}\left[#1\right]}}
\providecommand{\norm}[1]{\left\lVert#1\right\rVert}
\providecommand{\abs}[1]{\left\vert#1\right\vert}

\title{ \textbf{Math Computing}}
\author{ Chakali Suresh }
\date{}

\begin{document}
\vspace{-\baselineskip}
\maketitle

\section*{NCERT 9.7.1.7}

\textbf{This question is from class 9 ncert chapter 7.triangles}
\begin{enumerate}
	\item $\vec{AB}$ is a line segment and $\vec{P}$ is its mid-point. $\vec{D}$ and $\vec{E}$ are points on the same side of $\vec{AB}$ such that $\angle BAD = \angle ABE$ and $\angle EPA = \angle DPB$. Show that
		\begin{enumerate}
			\item $\triangle \vec{DAP} \cong  \triangle \vec{EBP}$
			\item $\vec{AD} = \vec{BE}$
\end{enumerate}
\begin{figure}[H]
    \includegraphics[width=\columnwidth]{figs/mc.png}
	\caption{$\triangle  \vec{DAP} \hspace{12pt} and \hspace{12pt} \triangle \vec{EBP}$}
 \label{fig:fig1}
\end{figure}
\pagebreak
\textbf{Construction steps:}
\\
		\begin{enumerate}[label=(\roman*)]
			\item Let point $\vec{A}$ be the reference point whose coordinates are at origin. 
\begin{align}
\vec{A} &= \myvec{0 \\ 0}
\end{align}

\item Let the distance between point $\vec{A}$ and $\vec{B}$ be $x$, and also considering the point $\vec{B}$ on same axis .
\begin{align}
	\norm{A-B} &= x
\end{align}
				So,the coordinates of point $\vec{B}$ be,
\begin{align}
\vec{B} &= \myvec{x \\ 0}
\end{align}

\item Given the point $\vec{P}$ is the mid-point of line segment $\vec{AB}$,
\begin{align}
	\vec{P} &= \myvec{\frac{A+B}{2}}\\
	\vec{P} &= \myvec{a \\ b}
\end{align}

\item Let the coordinate points of $\vec{D}$ and $\vec{E}$ are,
\begin{align}
\vec{D} &= \myvec{ x_1 \\ x_2},\\
\vec{E} &= \myvec{ x_3 \\ x_4}
\end{align}

\item Let assume the distance between point $\vec{A, D}$ and $\vec{B, E}$ be $\vec{r}$ ,and the line $\vec{AB}$ makes an angle $ \theta $ anticlock-wise from point $\vec{A}$ clock-wise from point $\vec{B}$ with the line $\vec{AD}$, $\vec{BE}$.
\begin{align}
	\norm{A-D} &= \vec{r} = \norm{B-E} \\
	\angle BAD &= \theta = \angle ABE
\end{align}

				$\therefore$ Now the coordinates of point $\vec{D, E}$ are,
\begin{align}
	\vec{D} &= \myvec{ x_1 \\ x_2 } = \myvec{r \cos \theta \\ r\sin \theta} \\
	\vec{E} &= \myvec{ x_3 \\ x_4 } = \myvec{-r \cos \theta \\ r \sin \theta} 
\end{align}

\item Similarly, the mid-point $\vec{P}$ also makes an angle $\theta$ with the points $\vec{D}$ and $\vec{E}$
	\begin{align}
		\angle BAD &= \theta = \angle ABE
	\end{align}

\item Let assume, 
\begin{table}[H]
\centering
	\begin{tabular}{|c|c|p{5cm}|}
\hline
\textbf{Symbol} & \textbf{Value} & \textbf{Description} \\
\hline
$\theta_1$ & $30\degree$ & $\angle{BAD} = \angle{ABE}$ \\
\hline
$\theta_2$ & $60\degree$ & $\angle{EPA} = \angle{DPB}$ \\
\hline
	$\vec{r}$ & 5 & Length of $\vec{AB}$ \\
\hline
	$\vec{A}$ & $\myvec{0\\0}$ & Reference point at origin \\
\hline
	$\vec{B}$ & $\myvec{5\\0}$ & point $\vec{B}$ on the same axis of $\vec{A}$ \\
\hline
	$\vec{P}$ & $\myvec{2.5\\0}$ & Mid-point of $\vec{AB}$ \\
\hline
	$\vec{D}$ & $\myvec{5 \cos 30\degree  \\ 5 \sin 30\degree}$ & From point $\vec{A}$ makes an angle $\theta_1$ in anticlock-wise with line $\vec{AB},\vec{AD}$  \\
\hline
$\vec{E}$ & $\myvec{-5 \cos 30\degree  \\ 5 \sin 30\degree}$ & From point $\vec{B}$ makes an angle $\theta_1$ in cloock-wise with line $\vec{AB},\vec{BE}$  \\
\hline 
$\vec{D}$ & $\myvec{5 \cos 60\degree  \\ 5 \sin 60\degree}$ & From point $\vec{P}$ makes an angle $\theta_2$ in anticlock-wise with line $\vec{BP},\vec{PD}$  \\
\hline
$\vec{E}$ & $\myvec{-5 \cos 60\degree  \\ 5 \sin 60\degree}$ & From point $\vec{P}$ makes an angle $\theta_2$ in clock-wise with line $\vec{AP},\vec{PE}$  \\
\hline

\end{tabular}

	  \caption{Parameters}
	  \label{Table}
\end{table}

\item on calculating we get,
\begin{align}
\vec{r} &= 5,\\
\vec{A} &= \myvec{0 \\ 0},\\
\vec{B} &= \myvec{5 \\ 0} ,\\
\vec{P} &= \myvec{2.5 \\ 0} ,\\
\vec{D} &= \myvec{ 4.330127 \\ 2.5 } ,\\
\vec{E} &= \myvec{-4.330127  \\ 2.5 },\\
\vec{D} &= \myvec{ 2.5 \\ 4.330127 } ,\\
\vec{E} &= \myvec{ -2.5 \\ 4.330127 }
\end{align}

Joining these points forms the required figure

\end{enumerate}
\end{enumerate}

\begin{figure}[H]
    \includegraphics[width=\columnwidth]{figs/fig_mat_comp.png}
	\caption{$\triangle \vec{DAP} \hspace{12pt} and \hspace{12pt} \triangle \vec{EBP}$}
    \label{fig:fig2}
\end{figure}

\end{document} 
